\documentclass[10pt]{scrartcl}
\usepackage[latin1]{inputenc}
\usepackage{bibgerm}
\usepackage[english, ngerman]{babel}
\usepackage[babel]{csquotes}
\usepackage{cite}
\usepackage{amsmath} % for eqref
\author{Denis Dietze \and Wolfgang Keller \and Nico Linke \and Thomas Schulte}

\title{Protokoll Versuch E10/IT7}
\subtitle{Mikroprozessor -- Grundlagenversuch Z 80 (Z 84 C 0008)}
\begin{document}
\maketitle
\section{Studienkontrollfragen}

\subsection{Maschinenzyklus}

\subsubsection{Frage}

Was ist ein Maschinenzyklus?

\subsubsection{Antwort}

Laut ~\cite[S.~27]{KieserMeder86} ist ein \emph{Maschinenzyklus} durch die Gesamtzahl der Takte einer Grundfunktion definiert. Eine oder mehrere Maschinenzyklen bilden dann einen \emph{Instruktionszyklus}. Grundfunktionen bilden:
\begin{itemize}
\item Befehlscode lesen
\item Speicher lesen
\item Speicher schreiben
\item I/O lesen
\item I/O schreiben
\item Interruptbest�tigung
\item CPU-interne Operation.
\end{itemize}

\subsection{Registerstruktur}

\subsubsection{Frage}

Beschreiben Sie die Registerstruktur der Z 80 CPU und erl�utern Sie die Besonderheiten und Verwendungsm�glichkeiten der Register!

\subsubsection{Antwort}

\subsection{Daten- vs. Befehlsbytes}

\subsubsection{Frage}

Wodurch werden Daten- und Befehlsbytes unterschieden?

\subsubsection{Antwort}

\subsection{Steuersignale}

\subsubsection{Frage}

Erl�utern Sie die Aufgaben der Steuersignale der Z 80 CPU!

\subsubsection{Antwort}

\subsection{Interrupt}

\subsubsection{Frage}

Was ist ein Interrupt?

\subsubsection{Antwort}

\subsection{Interruptmodi}

\subsubsection{Frage}

Erl�utern Sie die drei m�glichen Interruptmodi der Z 80 CPU!

\subsubsection{Antwort}

\subsection{Nichtmaskierbare vs. maskierbare Interrupts}

\subsubsection{Frage}

Erl�utern Sie den Unterschied zwischen nichtmaskierbaren Interrupts (NMI) und
einem maskierbaren Interrupt!

\subsubsection{Antwort}

\subsection{HALT-Zustand}

\subsubsection{Frage}

Wodurch ist der HALT-Zustand der Z 80 CPU gekennzeichnet?

\subsubsection{Antwort}

Laut ~\cite[S.~34]{KieserMeder86} ist der HALT-Zustand dadurch charakterisiert, dass w�hrend sich die CPU in diesem Zustand befindt, NOP-Befehle ausgef�hrt werden, deren Zweck der Refresh dynamischer Speicher ist.

Der HALT-Zustand kann von der CPU nur verlassen werden, wenn ein INT- oder NMI-Signal best"atigt wird.

\subsection{Adressbereich}

\subsubsection{Frage}

Wie gro� ist der Adressbereich f�r anschlie�bare Speicher und Ein-/Ausgabe-Tore
der Z 80 CPU?

\subsubsection{Antwort}

\subsection{Bidirektionaler Datentransfer}

\subsubsection{Frage}

Was bedeutet bidirektionaler Datentransfer?

\subsubsection{Antwort}

\section{Versuchsaufgaben}

\subsection{M1-Zyklus}

\subsubsection{Aufgabe}

Stellen Sie die f�r diesen Zyklus signifikanten Signale �ber den Taktimpulsen grafisch dar.
�berpr�fen Sie die Wirkung der WAIT-, RESET- und BUSRQ-Signale.

\subsubsection{L"osung}

\subsection{Speicher-Lese-/Schreib-Zyklus}

\subsubsection{Aufgabe}

Nehmen Sie entsprechend o. a. Muster das Zyklusdiagramm f�r einen Speicher-Lese-/Schreib-
Zyklus auf.

\subsubsection{L"osung}

\subsection{E/A-Zyklus}

\subsubsection{Aufgabe}

Programmieren Sie einen E/A-Zyklus und nehmen Sie das Zyklus-Diagramms auf. Halten Sie
die Inhalte des Adress- und Datenbusses schriftlich fest.

\subsubsection{L"osung}

\subsection{Interruptbehandlungszyklus}

\subsubsection{Aufgabe}

Nehmen Sie das Zyklusdiagramm auf und charakterisieren Sie die Unterschiede zu einem
normalen M1-Zyklus!

\paragraph{Hinweis} Vor Beginn ist das Interruptfreigabe-Flip-Flop durch einen entsprechenden Befehl
zu setzten!

\subsubsection{L"osung}

\subsection{NMI-Routine}

\subsubsection{Aufgabe}

Setzten Sie w�hrend eines M1-Zyklus ein NMI-Signal und halten Sie den weiteren Verlauf
fest!

\subsubsection{L"osung}

\subsection{HALT-Zustand}

Programmieren Sie einen Software-HALT und setzten Sie das Programm mit den vorhandenen
M�glichkeiten fort.

\subsubsection{Aufgabe}

\subsubsection{L"osung}

\subsection{RESTART-Befehl}

\subsubsection{Aufgabe}

Arbeiten Sie einen RST-38H-Befehl ab und halten Sie die Adress- und Datenbussignale fest.

\subsubsection{L"osung}

\subsection{Registeroperationen}

\subsubsection{Aufgabe}

F�hren Sie eine Operation mit den zug�nglichen Registern der Z 80 CPU nach eigener Wahl
aus.

\subsubsection{L"osung}

\bibliography{E10}{}
\bibliographystyle{geralpha}

\end{document}