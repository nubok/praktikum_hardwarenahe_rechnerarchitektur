\documentclass[10pt]{scrartcl}
\usepackage[latin1]{inputenc}
\usepackage[english, ngerman]{babel}
\usepackage[babel]{csquotes}
\usepackage{amsmath} % for eqref
\usepackage{tabularx}
\usepackage{graphicx}
\usepackage[usenames,dvipsnames]{color}
\usepackage{pstricks}
\usepackage{pst-node}
\usepackage{pstricks-add}
\usepackage{listings} % for listings - of course ;-)
\author{Denis Dietze\footnote{denis.dietze@st.ovgu.de} \and Wolfgang Keller\footnote{wolfgang.keller@student.uni-magdeburg.de} \and Nico Linke\footnote{nicolinke@googlemail.com} \and Thomas Schulte\footnote{thomas.schulte.md@googlemail.com}}
\title{Protokoll Versuch IT2}
\subtitle{Microcontroller Z8}
\begin{document}
\maketitle
\section{Zyklisches Ausl�sen der Analog-Digital-Umsetzung}
\subsection{Aufgabe}
\label{sec:aufg_3_2}
Entwickeln Sie eine Software, die zyklisch den ADU des USER-Boards �ber das Signal $\overline{\textnormal{CEN}}$ (Conversion Enable) startet und zun�chst die Spannung an Kanal 1 (CH1) in einen Digitalwert wandelt. Danach testen Sie die Software, modifiziert zur Erfassung der Spannung am Kanal 2 (siehe Anlage 2/3 der Versuchsanleitung).

Geben Sie zus�tzlich die oberen 4-Bit der ADU-Daten auf die LED-Balkenanzeige (arbeitet bin�r kodiert) aus (siehe Anlage 2/3 der Versuchsanleitung).

\paragraph{Anmerkung 1} Mit jeder Low/High-Flanke des Signals $\overline{\textnormal{CEN}}$ (P35) werden automatisch die bin�ren ADU-Daten (vorzeichenlose Darstellung) in das LED-Modul (Leuchtdioden $2^0$ \ldots $2^7$) eingeschrieben und somit sichtbar gemacht.

\paragraph{Anmerkung 2} Port 2 als Ausgabe-Port mu� mit aktiven 'pull ups' betrieben werden. Dazu muss Steuerregister P3M entsprechend konfiguriert werden!

\subsection{L"osung}

\begin{verbatim}
	TITLE "Aufgabe 3.2"
	ORG %0C
	
	; Initialisierung
	
	; Laut Anlage 2 ist der ADU P35, P34 (fuer Controller
	; Ausgabe) und P30 (fuer Z8 Eingabe) angeschlossen
	; 3M ist somit zu setzen
	; Bit 0: 1 (da Push-Pull)
	;     1: 0 Reserved
	;     2: 0 P35=Output
	;  3, 4: 00 (da P34 Output)
	;     5: sollte egal; P sein (0)
	;     6: 0 (P30 Input)
	;     7: 0 (Parity Off)
	; 00000001 = #%01
	LD P3M, #%01
	
	; Laut Anlage 2 ist die Ausgabe des ADU-Moduls in P00-P07
	; P01M ist somit zu setzen
	; Bit 1, 0: 01 (P0_3-P0_0 Input)
	;        2: egal (0)
	;     4, 3: egal (00)
	;        5: 0 (normales Timing)
	;     7, 6: 01 (P0_7-P0_4 Input)
	; 01000001 = #%41
	LD P01M, #%41
	
	; Laut Anlage 2 ist das Balkenanzeige-Modul an P27-P24
	; angeschlossen
	; P2M ist somit zu setzen
	; 00001111 (oberste 4 Bits Output) = #%0F
	LD P2M, #%0F
	
loop:
	; ADU starten
	
	; An P3:
	; P34 (Channel Select) = 0 (Channel 1)
	;                 oder = 1 (Channel 2)
	; P35 (not CEN)        = 0
	; restliche Bits 0
	LD P3, #%00 ; oder #%10 (Channel 2)
	
	NOP         ; (siehe Anlage 3)
	
	; Laut Anlage 1 befindet sich das Ergebnis der
	; A-D-Umwandlung in Port 0
	; Wir koennen wir einfach direkt LD P2, 4 benutzen (da wir die unteren 4
	; Bits von Port 2 als Input deklariert haben)
	
	LD P2, P0

	jmp loop
\end{verbatim}

\section{Bin"ar/BCD-Konvertierung f�r 7-Segment-Anzeige}
\subsection{Aufgabe}

Entwickeln Sie eine Software, die zyklisch den ADU startet und das Ergebnis-Byte der AD-Umsetzung (Daten D0 \ldots D7) in einen 3-stelligen BCD-Code wandelt und �ber das 7-Segment-Anzeigemodul ausgibt (siehe Anlage 2/3 der Versuchsanleitung).

Hierbei soll zun�chst nur die Spannung am Kanal 1 des ADU ausgewertet werden.

Geben Sie auch hierbei die oberen 4-Bit der ADU-Daten auf die LED-Balkenanzeige aus
(P24 ... P27).

\paragraph{Anmerkung} Der ADU ist so kalibriert, dass er eine unipolare Eingangangsspannung von maximal 25,5 V verarbeiten kann. Da diese Spannung auf einen Digitalwert von 255 (FFh) abgebildet wird, ist somit \emph{keine} spezielle Skalierung vor der Bin�r/BCD-Konvertierung erforderlich.

\subsection{L"osung}

Sei Register 4 das Register, welches das Ergebnis-Byte der AD-Umsetzung beinhaltet. Die Register 5 und das untere Nibble von Register 6 sollen den BSD-codierten Wert beinhalten. Als Register f"ur Zwischenwerte werden au�erdem die Register 10, 11, 12

\begin{verbatim}
	TITLE "Aufgabe 3.3"
	ORG %0C
	
	; Initialisierung
	
	; Laut Anlage 2 ist der ADU P35, P34 (fuer Controller
	; Ausgabe) und P30 (fuer Z8 Eingabe) angeschlossen
	; 3M ist somit zu setzen
	; Bit 0: 1 (da Push-Pull)
	;     1: 0 Reserved
	;     2: 0 P35=Output
	;  3, 4: 00 (da P34 Output)
	;     5: sollte egal; P sein (0)
	;     6: 0 (P30 Input)
	;     7: 0 (Parity Off)
	; 00000001 = #%01
	LD P3M, #%01
	
	; Laut Anlage 2 ist die Ausgabe des ADU-Moduls in P00-P07
	; P01M ist somit zu setzen
	; Bit 1, 0: 01 (P0_3-P0_0 Input)
	;        2: 0 (externen Stack anlegen)
	;     4, 3: 01 (P1_7-P1_0 Byte Output)
	;        5: 0 (normales Timing)
	;     7, 6: 01 (P0_7-P0_4 Input)
	; 01001001 = #%49
	LD P01M, #%49
	
	; Laut Anlage 2 ist die 7-Segment-Anzeige au�erdem an P23-P20
	; angeschlossen
	; P2M ist somit zu setzen
	; 11110000 (unterste 4 Bits Output) = #%F0
	LD P2M, #%F0
	
	; Stack Pointer initialisieren (vgl. Versuch IT2)
	LD SPH,  #%11
	LD SPL,  #%00
	
loop:
	; ADU starten
	
	; An P3:
	; P34 (Channel Select) = 0 (Channel 1)
	;                 oder = 1 (Channel 2)
	; P35 (not CEN)        = 0
	; restliche Bits 0
	LD P3, #%00 ; oder #%10 (Channel 2)
	
	NOP         ; (siehe Anlage 3)
	
	; In Register 4 soll der ausgelesene Wert liegen
	LD 4, P0
	
	; BCD-Konvertierung in Register 5 und 6
	
	LD 10, 4
	CALL begin_routine
	LD 5, 11           ; Register 5 enthaelt kleinstes Nibble des BCD-Wertes

	LD 10, 12
	CALL begin_routine
	LD 6, 11
	
	; Register 6 um 4 Bits nach oben
	; shiften = 4 * nach rechts rotieren
	; + nun untere 4 Bits auf 0 setzen
	RL 6
	RL 6
	RL 6
	RL 6
	AND 6, #F0
	
	; Nun diese 4 Bits nach Register 5 bringen
	OR 5, 6
	
	LD 10, 12
	CALL moddiv10
	LD 6, 11
	
	; Ausgabe
	LD P1, 5
	LD P2, 6
	
	JMP loop
	
	
	
	; Routine, die Byte in Register 10 annimmt
	; und in Register 11 den Modulus  bzgl. 10
	; und in Register 12 die Division bzgl. 10
	; ablegt
	;
	; Keines der Register bleibt unveraendert
moddiv10:
	LD 11, 0
	LD 12, 0
begin_loop_routine:
	CP 10, #9          ; Vergleiche Wert mit 9
	JP MI, end_loop0
	SUB 10, #10
	INC 12             ; Wir ziehen #10 ab - also Register 12 inkrementieren
	JP begin_loop_routine
end_loop0: 
	LD 11, 10
	RET
\end{verbatim}

\section{Zeitmultiplexbetrieb der ADU-Kan�le}
\subsection{Aufgabe}

Erweitern Sie die Software von Aufgabe \ref{sec:aufg_3_2} derart, dass Sie im Zeitmultiplexverfahren beide Kanaleingangsspannungen des ADU digitalisieren. Jeder Kanal sollte dabei f�r ca. 4 s aktiviert werden (Software-Zeitschleife generieren).

\subsection{L"osung}

\begin{verbatim}
	TITLE "Aufgabe 3.4"
	ORG %0C
	
	; Initialisierung
	
	; Laut Anlage 2 ist der ADU P35, P34 (fuer Controller
	; Ausgabe) und P30 (fuer Z8 Eingabe) angeschlossen
	; 3M ist somit zu setzen
	; Bit 0: 1 (da Push-Pull)
	;     1: 0 Reserved
	;     2: 0 P35=Output
	;  3, 4: 00 (da P34 Output)
	;     5: sollte egal; P sein (0)
	;     6: 0 (P30 Input)
	;     7: 0 (Parity Off)
	; 00000001 = #%01
	LD P3M, #%01
	
	; Laut Anlage 2 ist die Ausgabe des ADU-Moduls in P00-P07
	; P01M ist somit zu setzen
	; Bit 1, 0: 01 (P0_3-P0_0 Input)
	;        2: 0 (externen Stack anlegen)
	;     4, 3: 01 (P1_7-P1_0 Byte Output)
	;        5: 0 (normales Timing)
	;     7, 6: 01 (P0_7-P0_4 Input)
	; 01001001 = #%49
	LD P01M, #%49
	
	; Laut Anlage 2 ist die 7-Segment-Anzeige au�erdem an P23-P20
	; angeschlossen
	; P2M ist somit zu setzen
	; 11110000 (unterste 4 Bits Output) = #%F0
	LD P2M, #%F0
	
	; Stack Pointer initialisieren (vgl. Versuch IT2)
	LD SPH,  #%11
	LD SPL,  #%00

	; Register Pointer aktivieren (fuer Schleifen)	
	LD RP, #%10
	
loop:
	LD 10, #%00
	CALL channel_read_and_output_loop
	
	LD 10, #%10
	CALL channel_read_and_output_loop
	
	JMP loop
	
	; Routine, die Byte in Register 10 annimmt
	; und in Register 11 den Modulus  bzgl. 10
	; und in Register 12 die Division bzgl. 10
	; ablegt
	;
	; Keines der Register bleibt unveraendert
moddiv10:
	LD 11, 0
	LD 12, 0
begin_loop_routine:
	CP 10, #9          ; Vergleiche Wert mit 9
	JP MI, end_loop0
	SUB 10, #10
	INC 12             ; Wir ziehen #10 ab - also Register 12 inkrementieren
	JP begin_loop_routine
end_loop0: 
	LD 11, 10
	RET
	
	
	
	; Routine, die zu initialisierenden Wert fuer P3 in Register 10
	; (#%00 fuer Channel 1 und #%10 fuer Channel 2) entgegennimmt und diesen
	; auf der 7-Segment-Anzeige ausgibt.
channel_read_and_output:
	; ADU starten
	LD P3, 10
	
	NOP         ; (siehe Anlage 3)
	
	; In Register 4 soll der ausgelesene Wert liegen
	LD 4, P0
	
	; BCD-Konvertierung in Register 5 und 6
	
	LD 10, 4
	CALL begin_routine
	LD 5, 11           ; Register 5 enthaelt kleinstes Nibble des BCD-Wertes

	LD 10, 12
	CALL begin_routine
	LD 6, 11
	
	; Register 6 um 4 Bits nach oben
	; shiften = 4 * nach rechts rotieren
	; + nun untere 4 Bits auf 0 setzen
	RL 6
	RL 6
	RL 6
	RL 6
	AND 6, #F0
	
	; Nun diese 4 Bits nach Register 5 bringen
	OR 5, 6
	
	LD 10, 12
	CALL moddiv10
	LD 6, 11

	; Ausgabe	
	LD P1, 5
	LD P2, 6
	
	RET



	; Routine, die in einer Schleife fuer ca. 4 Sekunden
	; den Port abfragt. Parameter: in Register 10 zu initialisierender
	; Wert fuer P3
channel_read_and_output_loop:
	; Schleifen
	LD R0, #%FF
loop0:
	LD R1, #%FF
loop1:
	LD R2, #%FF
loop2:
	LD R3, #%FF
loop3:
	CALL channel_read_and_output
	DJNZ R3, loop3
	DJNZ R2, loop2
	DJNZ R1, loop1
	DJNZ R0, loop0	
	
	RET
\end{verbatim}

\end{document}