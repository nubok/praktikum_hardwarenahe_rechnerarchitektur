\documentclass[10pt]{scrartcl}
\usepackage[latin1]{inputenc}
\usepackage[english, ngerman]{babel}
\usepackage[babel]{csquotes}
\usepackage{amsmath} % for eqref
\usepackage{tabularx}
\usepackage{graphicx}
\usepackage[usenames,dvipsnames]{color}
\usepackage{pstricks}
\usepackage{pst-node}
\usepackage{pstricks-add}
\usepackage{listings} % for listings - of course ;-)
\author{Denis Dietze\footnote{denis.dietze@st.ovgu.de} \and Wolfgang Keller\footnote{wolfgang.keller@student.uni-magdeburg.de} \and Nico Linke\footnote{nicolinke@googlemail.com} \and Thomas Schulte\footnote{thomas.schulte.md@googlemail.com}}
\title{Protokoll Versuch IT2}
\subtitle{Microcontroller Z8}
\begin{document}
\maketitle
\section{Zyklisches Ausl�sen der Analog-Digital-Umsetzung}
\subsection{Aufgabe}
\label{sec:aufg_3_2}
Entwickeln Sie eine Software, die zyklisch den ADU des USER-Boards �ber das Signal $\overline{\textnormal{CEN}}$ (Conversion Enable) startet und zun�chst die Spannung an Kanal 1 (CH1) in einen Digitalwert wandelt. Danach testen Sie die Software, modifiziert zur Erfassung der Spannung am Kanal 2 (siehe Anlage 2/3 der Versuchsanleitung).

Geben Sie zus�tzlich die oberen 4-Bit der ADU-Daten auf die LED-Balkenanzeige (arbeitet bin�r kodiert) aus (siehe Anlage 2/3 der Versuchsanleitung).

\paragraph{Anmerkung 1} Mit jeder Low/High-Flanke des Signals $\overline{\textnormal{CEN}}$ (P35) werden automatisch die bin�ren ADU-Daten (vorzeichenlose Darstellung) in das LED-Modul (Leuchtdioden $2^0$ \ldots $2^7$) eingeschrieben und somit sichtbar gemacht.

\paragraph{Anmerkung 2} Port 2 als Ausgabe-Port mu� mit aktiven 'pull ups' betrieben werden. Dazu muss Steuerregister P3M entsprechend konfiguriert werden!

\subsection{L"osung}



\section{Bin"ar/BCD-Konvertierung f�r 7-Segment-Anzeige}
\subsection{Aufgabe}

Entwickeln Sie eine Software, die zyklisch den ADU startet und das Ergebnis-Byte der AD-Umsetzung (Daten D0 \ldots D7) in einen 3-stelligen BCD-Code wandelt und �ber das 7-Segment-Anzeigemodul ausgibt (siehe Anlage 2/3 der Versuchsanleitung).

Hierbei soll zun�chst nur die Spannung am Kanal 1 des ADU ausgewertet werden.

Geben Sie auch hierbei die oberen 4-Bit der ADU-Daten auf die LED-Balkenanzeige aus
(P24 ... P27).

\paragraph{Anmerkung} Der ADU ist so kalibriert, dass er eine unipolare Eingangangsspannung von maximal 25,5 V verarbeiten kann. Da diese Spannung auf einen Digitalwert von 255 (FFh) abgebildet wird, ist somit \emph{keine} spezielle Skalierung vor der Bin�r/ BCD-Konvertierung erforderlich.

\subsection{L"osung}

\section{Zeitmultiplexbetrieb der ADU-Kan�le}
\subsection{Aufgabe}

Erweitern Sie die Software von Aufgabe \ref{sec:aufg_3_2} derart, dass Sie im Zeitmultiplexverfahren beide Kanaleingangsspannungen des ADU digitalisieren. Jeder Kanal sollte dabei f�r ca. 4 s aktiviert werden (Software-Zeitschleife generieren).

\subsection{L"osung}


\end{document}