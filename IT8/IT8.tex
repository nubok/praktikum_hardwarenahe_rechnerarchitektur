\documentclass[10pt]{scrartcl}
\usepackage[latin1]{inputenc}
\usepackage[english, ngerman]{babel}
\usepackage[babel]{csquotes}
\usepackage{amsmath} % for eqref
\usepackage{tabularx}
\usepackage{graphicx}
\usepackage[usenames,dvipsnames]{color}
\usepackage{pstricks}
\usepackage{pst-node}
\usepackage{pstricks-add}
\usepackage{listings} % for listings - of course ;-)
\usepackage{float}
\author{Denis Dietze\footnote{denis.dietze@st.ovgu.de} \and Wolfgang Keller\footnote{wolfgang.keller@student.uni-magdeburg.de} \and Nico Linke\footnote{nicolinke@googlemail.com} \and Thomas Schulte\footnote{thomas.schulte.md@googlemail.com}}
\title{Protokoll Versuch IT8}
\subtitle{Microcontroller Z8 mit ADU-Interface}
\begin{document}
\maketitle
\tableofcontents

\lstdefinelanguage{z8asm}
{
morekeywords={ORG,AND,CALL,CP,DJNZ,INC,JP,LD,LDE,NOP,OR,RET,RLC,RRC,SUB,TITLE,END},
morecomment=[l]{;}
}

\lstset{
numbers=left,                   % where to put the line-numbers,
commentstyle=\color{OliveGreen},
captionpos=b, % Ist t oder b besser?
showtabs=true,                 % show tabs within strings adding particular underscores
frame=single,	                % adds a frame around the code
tabsize=2	                % sets default tabsize to 2 spaces
}

\section{Vorliegende Situation des Experiments}

\begin{figure}
\begin{center}
\begin{pspicture}(17,5)
%\psgrid
\begin{psmatrix}[rowsep=0.4,colsep=1.0]
	~ & ~ & \psframebox{\parbox[c][0.7cm][c]{2.15cm}{\centering Bin"ar nach BCD}} & \psframebox{\parbox[c][0.7cm][c]{2.15cm}{\centering 7-Segment-Decoder}} & \psframebox{\parbox[c][0.7cm][c]{2.15cm}{\centering Anzeigen $10^1$, $10^0$, $10^{-1}$}} \\
	~ & \psframebox{\parbox[c][0.7cm][c]{1.2cm}{\centering ADU}} & \parbox{0cm}{} & ~ & \psframebox{\parbox[c][0.7cm][c]{2.15cm}{\centering 8 LEDs}}  \\
	~ & ~ & \psframebox{\parbox[c][0.7cm][c]{2.15cm}{\centering Oberes Nibble}} & \psframebox{\parbox[c][0.7cm][c]{2.15cm}{\centering Decoder}} & \psframebox{\parbox[c][0.7cm][c]{2.15cm}{\centering 15 LEDs}}
\end{psmatrix}
\ncline[offset=0.4cm]{->}{1,3}{1,4}\ncput*{4}
\ncline{->}{1,3}{1,4}\ncput*{4}
\ncline[offset=-0.4cm]{->}{1,3}{1,4}\ncput*{4}
\ncline[offset=0.4cm]{->}{1,4}{1,5}\ncput*{7}
\ncline{->}{1,4}{1,5}\ncput*{7}
\ncline[offset=-0.4cm]{->}{1,4}{1,5}\ncput*{7}
\ncline{->}{2,1}{2,2}\ncput*{1}\nbput{$U_E$}
\ncline{-}{2,2}{2,3}\naput{Digits}\nbput{$\{0, 1\}$}
\ncline{->}{2,3}{2,5}\ncput*{8}
\ncline{->}{2,3}{1,3}\ncput*{8}
\ncline{->}{2,3}{3,3}\ncput*{8}
\ncline{->}{3,3}{3,4}\ncput*{4}
\ncline{->}{3,4}{3,5}\ncput*{15}
\end{pspicture}
\end{center}
\caption{Im Experiment vorliegende Situation}
\label{abb:situation}
\end{figure}

Abbildung \ref{abb:situation} stellt die allgemeine Situation des Experiments dar: eine Eingangsspannung $U_E$ wird durch den ADU in 8 $\{0, 1\}$-wertige Digits umgesetzt. Diese werden als eine Verarbeitungsm"oglichkeit direkt auf den LEDs angezeigt (mittlere Zeile der Abbildung). 

Eine weitere Option stellt das "Ubernehmen der oberen vier Bits (oberes Nibble) dar, um dieses an einen Decoder (welcher sich als Baustein auf der Platine befindet) zu senden, der die vier Bits auf 15 Bits decodiert. Diese sind an 15 LEDs angeschlossen, die als Balkenanzeige die Signalst"arke visualisieren. Das von uns zu entwickelnde Programm f"ur den mit "`Oberes Nibble"' bezeichneten Decoder wird in Abschnitt \ref{sec:3_2} beschrieben.

Eine dritte M"oglichkeit, die im Rahmen des Experiments behandelt wird, ist die BCD-Codierung des Bytes, wof"ur in Abschnitt \ref{sec:3_3} ein Programm entwickelt wird. Das Ergebnis wird anschlie"send durch einen auf der Platine aufgel"oteteten Sieben-Segment-Decoder in drei mal sieben Bits (f"ur jedes Segment ein Bit) "uberf"uhrt, welche auf drei Sieben-Segment-Anzeigen dargestellt werden und somit f"ur eine numerische Darstellung der Eingangsspannung sorgen.


\section{Aufgaben und L"osungen}
\subsection{Zyklisches Ausl�sen der Analog-Digital-Umsetzung}
\label{sec:3_2}
\subsubsection{Aufgabe}
\label{sec:aufg_3_2}
Entwickeln Sie eine Software, die zyklisch den ADU des USER-Boards �ber das Signal $\overline{\textnormal{CEN}}$ (Conversion Enable) startet und zun�chst die Spannung an Kanal 1 (CH1) in einen Digitalwert wandelt. Danach testen Sie die Software, modifiziert zur Erfassung der Spannung am Kanal 2 (siehe Anlage 2/3 der Versuchsanleitung).

Geben Sie zus�tzlich die oberen 4-Bit der ADU-Daten auf die LED-Balkenanzeige (arbeitet bin�r kodiert) aus (siehe Anlage 2/3 der Versuchsanleitung).

\paragraph{Anmerkung 1} Mit jeder Low/High-Flanke des Signals $\overline{\textnormal{CEN}}$ (P35) werden automatisch die bin�ren ADU-Daten (vorzeichenlose Darstellung) in das LED-Modul (Leuchtdioden $2^0$ \ldots $2^7$) eingeschrieben und somit sichtbar gemacht.

\paragraph{Anmerkung 2} Port 2 als Ausgabe-Port mu� mit aktiven 'pull ups' betrieben werden. Dazu muss Steuerregister P3M entsprechend konfiguriert werden!

\subsubsection{L"osung}
\label{sec:lsg_3_2}

Um die Plausibilit"at der Ausgaben des Programms einfacher analysieren zu k"onnen, erstellten wir eine Tabelle mit Eingangsspannungen $U_E$ und erwarteten Ausgaben des ADUs. In Tabelle \ref{table:aufgabe32} sind diese Werte dokumentiert.

Prinzipiell k"onnte man - sofern eine hinreichend stabile und genaue Spannungsquelle vorliegt - diese Referenzwerte auch zur Kalibrierung des ADUs benutzen, beispielsweise indem man mittels linearer Regression eine Ausgleichsgerade bestimmt. Aus offensichtlichen Gr"unden wurde jedoch im Rahmen dieses Experiments darauf nicht weiter eingegangen.

\begin{table}
\begin{tabular}[t]{|r|r|}
\hline
Eingangsspannung $U_E$ & Digitaler Wert \\
\hline
$0,0~V$ & \verb|00000000| \\
$1,5~V$ & \verb|00001111| \\
$1,6~V$ & \verb|00010000| \\
$3,1~V$ & \verb|00011111| \\
$3,2~V$ & \verb|00100000| \\
$4,7~V$ & \verb|00101111| \\
$4,8~V$ & \verb|00110000| \\
$6,3~V$ & \verb|00111111| \\
$6,4~V$ & \verb|01000000| \\
$7,9~V$ & \verb|01001111| \\
$8,0~V$ & \verb|01010000| \\
$9,5~V$ & \verb|01011111| \\
$9,6~V$ & \verb|01100000| \\
$11,1~V$ & \verb|01101111| \\
$11,2~V$ & \verb|01110000| \\
$12,7~V$ & \verb|01111111| \\
$12,8~V$ & \verb|10000000| \\
$14,3~V$ & \verb|10001111| \\
$14,4~V$ & \verb|10010000| \\
$15,9~V$ & \verb|10011111| \\
$16,0~V$ & \verb|10100000| \\
$17,5~V$ & \verb|10101111| \\
$17,6~V$ & \verb|10110000| \\
$19,1~V$ & \verb|10111111| \\
$19,2~V$ & \verb|11000000| \\
$20,7~V$ & \verb|11001111| \\
$20,8~V$ & \verb|11010000| \\
$22,3~V$ & \verb|11011111| \\
$22,4~V$ & \verb|11100000| \\
$23,9~V$ & \verb|11101111| \\
$24,0~V$ & \verb|11110000| \\
$25,5~V$ & \verb|11111111| \\
\hline
\end{tabular}
\caption{Erwartete Werte der AD-Umwandlung bei verschiedenen Eingangsspannungen $U_E$}
\label{table:aufgabe32}
\end{table}

Nun zum eigentlichen Programm: in Abbildung \ref{abb:aufgabe32} ist befindet sich das Flussdiagramm des zu erstellenden Programms. Listing \label{lsg:aufgabe32} zeigt das daraus erstellte Assembler-Programm.

\begin{figure}
\begin{center}
\begin{pspicture}(5,19)
%\psgrid
\begin{psmatrix}[rowsep=0.4,colsep=0.5]
	\psovalbox{Begin} \\
	~ & [name=Init1] & \makebox[5cm]{} & [name=Init1a]\\
	\psframebox{\parbox{4cm}{45h in P01M schreiben}}  \\
	~ & [name=Init2] \\
	\psframebox{\parbox{4cm}{00h in P2M schreiben}} \\
	~ & [name=Init3] \\
	\psframebox{\parbox{4cm}{01h auf P3M schreiben}} \\
	~ & [name=Init4] \\
	\psframebox{\parbox{4cm}{RP mit 10h initialisieren}} \\
	~ & [name=Init5] & \makebox[5cm]{} & [name=Init5a] \\
	\psframebox{\parbox{4cm}{P3 auf 20h (Benutzung Channel 1) bzw. 30h (Benutzung Channel 2) setzen}} \\
	\psframebox{\parbox{4cm}{P3 auf 00h (Benutzung Channel 1) bzw. 10h (Benutzung Channel 2) setzen}} \\
	\psframebox{\parbox{4cm}{Einen Zyklus abwarten}} \\
	~ & [name=Loop1] \\
	\psframebox{\parbox{4cm}{Wert von P0 nach R0 laden}} \\
	\psframebox{\parbox{4cm}{Untere 4 Bits von R0 auf 1 setzen}} \\
	\psframebox{\parbox{4cm}{Wert von R0 in P2 ablegen}} \\
	\psframebox{\parbox{4cm}{FFh in P1 speichern}} \\
	~ & [name=Loop2] & \makebox[5cm]{} & [name=Loop2a]
\end{psmatrix}
\ncline{->}{1,1}{3,1}
\ncline{->}{3,1}{5,1}
\ncline{->}{5,1}{7,1}
\ncline{->}{7,1}{9,1}
\ncline{->}{9,1}{11,1}
\ncline{->}{11,1}{12,1}
\ncline{->}{12,1}{13,1}
\ncline{->}{13,1}{15,1}
\ncline{->}{15,1}{16,1}
\ncline{->}{16,1}{17,1}
\ncline{->}{17,1}{18,1}
\ncangles[angleA=-90, armA=0.5cm, armB=2.5cm]{->}{18,1}{10,1}
\psbrace[ref=lC](Init2)(Init1){\parbox{5cm}{Port 0 auf Input und Port 1 auf Output konfigurieren}}
\psbrace[ref=lC](Init3)(Init2){\parbox{5cm}{Alle Bits von Port 2 als Output konfigurieren}}
\psbrace[ref=lC](Init4)(Init3){\parbox{5cm}{In Port 2 'push-pull' aktivieren, Port 3 f"ur ADU konfigurieren}}
\psbrace[ref=lC](Init5)(Init4){\parbox{5cm}{Initialisierung Register Pointer}}
\psbrace[ref=lC](Init5a)(Init1a){\parbox{3cm}{Initialisierung}}
\psbrace[ref=lC](Loop1)(Init5){\parbox{5cm}{A-D-Umwandlung ausl"osen}}
\psbrace[ref=lC](Loop2)(Loop1){\parbox{5cm}{Digitalisiertes Datum auslesen, verarbeiten und ausgeben}}
\psbrace[ref=lC](Loop2a)(Init5a){\parbox{3cm}{Hauptschleife}}
\end{pspicture}
\end{center}
\caption{Zyklisches Ausl�sen der Analog-Digital-Umsetzung}
\label{abb:aufgabe32}
\end{figure}

\begin{figure}
\lstinputlisting[caption={Assembler-Programm f"ur zyklisches Ausl�sen der Analog-Digital-Umsetzung}, language=z8asm, label=lsg:aufgabe32]{aufgabe32.asm}
\end{figure}

Prinzipiell sollte der Ablauf des Programms relativ klar sein. Der einzige etwas kompliziertere Teil ist die Verarbeitung und Ausgabe des digitalisieren Werts. 

Der Grund, warum wir die unteren vier Bits von P0 auf 1 setzen und nach P1 den Wert FFh senden, liegt darin begr"undet, dass gem"a"s Anlage 2 der Versuchsbeschreibung das Sieben-Segment-Anzeige-Modul an den unteren 4 Bits von Port 2 und dem gesamten Port 1 angeschlossen ist. Da es unerw"unscht ist, dass in diesem Programm die Sieben-Segment-Anzeige etwas anzeigt, schreiben wir in die vier Bits der Steuerbits jeder der vier Ziffern den Wert FFh (also einen Wert au�erhalb des Intervalls von 0 bis 9), damit keines der Segmente aufleuchtet.

\clearpage

\subsection{Bin"ar/BCD-Konvertierung f�r 7-Segment-Anzeige}
\label{sec:3_3}
\subsubsection{Aufgabe}

Entwickeln Sie eine Software, die zyklisch den ADU startet und das Ergebnis-Byte der AD-Umsetzung (Daten D0 \ldots D7) in einen 3-stelligen BCD-Code wandelt und �ber das 7-Segment-Anzeigemodul ausgibt (siehe Anlage 2/3 der Versuchsanleitung).

Hierbei soll zun�chst nur die Spannung am Kanal 1 des ADU ausgewertet werden.

Geben Sie auch hierbei die oberen 4-Bit der ADU-Daten auf die LED-Balkenanzeige aus
(P24 ... P27).

\paragraph{Anmerkung} Der ADU ist so kalibriert, dass er eine unipolare Eingangangsspannung von maximal 25,5 V verarbeiten kann. Da diese Spannung auf einen Digitalwert von 255 (FFh) abgebildet wird, ist somit \emph{keine} spezielle Skalierung vor der Bin�r/BCD-Konvertierung erforderlich.

\subsubsection{L"osung}

Gem"a"s Anweisungen des Betreuers soll dieses Programm durch Erweiterung des Programms der vorhergehenden Aufgabe entwickelt werden.

Zuerst betrachten wir die Initialisierung. Sie erfolgt wie bei dem Programm in Abschnitt \ref{sec:lsg_3_2}, nur dass zus"atzlich ein Stack initialisiert werden muss. Dies erfolgt dadurch, dass wir SPL auf 80h setzen. Somit liegt der Stack in einem Bereich in der Mitte der Register, wo es zu keiner Kollision mit von der CPU (vgl. dazu Anlage 6 der Versuchsanleitung) oder im Programm verwendeten Registern kommt.

Abbildung \ref{abb:aufgabe33_1} bildet das Flussdiagramm des Programms unter Detailbetrachtung der Initialisierung und Listing \ref{lsg:aufgabe33_1} zeigt den Assembler-Code zur Initialisierung.

\begin{figure}
\begin{center}
\begin{pspicture}(5,11)
%\psgrid
\begin{psmatrix}[rowsep=0.4,colsep=0.5]
	\psovalbox{Begin} \\
	~ & [name=Init1] & \makebox[5cm]{} & [name=Init1a]\\
	\psframebox{\parbox{4cm}{45h in P01M schreiben}}  \\
	~ & [name=Init2] \\
	\psframebox{\parbox{4cm}{00h in P2M schreiben}} \\
	~ & [name=Init3] \\
	\psframebox{\parbox{4cm}{01h auf P3M schreiben}} \\
	~ & [name=Init4] \\
	\psframebox{\parbox{4cm}{RP mit 10h initialisieren}} \\
	~ & [name=Init5] \\
	\psframebox{\parbox{4cm}{SPL auf 80h setzen}} \\
	~ & [name=Init6] & \makebox[5cm]{} & [name=Init6a] \\
	\psframebox{\parbox{4cm}{Hauptschleife}}
\end{psmatrix}
\ncline{->}{1,1}{3,1}
\ncline{->}{3,1}{5,1}
\ncline{->}{5,1}{7,1}
\ncline{->}{7,1}{9,1}
\ncline{->}{9,1}{11,1}
\ncline{->}{11,1}{13,1}
\psbrace[ref=lC](Init2)(Init1){\parbox{5cm}{Port 0 auf Input und Port 1 auf Output konfigurieren}}
\psbrace[ref=lC](Init3)(Init2){\parbox{5cm}{Alle Bits von Port 2 als Output konfigurieren}}
\psbrace[ref=lC](Init4)(Init3){\parbox{5cm}{In Port 2 'push-pull' aktivieren, Port 3 f"ur ADU konfigurieren}}
\psbrace[ref=lC](Init5)(Init4){\parbox{5cm}{Initialisierung Register Pointer}}
\psbrace[ref=lC](Init6)(Init5){\parbox{5cm}{Initialisierung Stack Pointer}}
\psbrace[ref=lC](Init6a)(Init1a){\parbox{3cm}{Initialisierung}}
\end{pspicture}
\end{center}
\caption{Bin"ar/BCD-Konvertierung f�r 7-Segment-Anzeige -- Detailbetrachtung Initialisierung}
\label{abb:aufgabe33_1}
\end{figure}

\begin{figure}
\lstinputlisting[caption={Assembler-Programm f"ur Bin"ar/BCD-Konvertierung f�r 7-Segment-Anzeige -- Initialisierung}, language=z8asm, label=lsg:aufgabe33_1, lastline=12]{aufgabe33.asm}
\end{figure}

Als n"achtes betrachten wir zur Hauptschleife. Abbildung \ref{abb:aufgabe33_2} zeigt ein Flussdiagramm des Programms, welches die Hauptschleife im Detail betrachtet. Der Assembler-Code der Hauptschleife befindet sich in Listing \ref{lsg:aufgabe33_2}. Da das Programm etwas umfangreicher ist, befindet sich in Tabelle \ref{table:aufgabe33} eine Registerbelegungstabelle des gesamten Programms dieser Aufgabe.

Die Grundidee hinter der L"osung ist es, drei mal nacheinander die Funktion "`moddiv10"' aufzurufen, welche den Modulus und das Ergebnis der ganzzahligen Division bez"uglich 10 bestimmt. Der Modulus wird als aktuelle BCD-Tetrade "ubernommen und das zwei mal mit dem Ergebnis der ganzzahligen Division durch 10 wiederholt.

\begin{figure}
\begin{center}
\begin{pspicture}(5,21.5)
%\psgrid
\begin{psmatrix}[rowsep=0.4,colsep=0.5]
	\psovalbox{Begin} \\
	\psframebox{\parbox{4cm}{Initialisierung}}  \\
	~ & [name=Init5] & \makebox[5cm]{} & [name=Init5a] \\
	\psframebox{\parbox{4cm}{A-D-Umwandlung ausl"osen und Ergebnis nach R0 auslesen}} \\
	~ & [name=Tetrade0] \\
	\psframebox{\parbox{4cm}{Rufe Funktion "`moddiv10"' mit Parameter R0 auf und speichere Modulus in Register 5}} \\
	~ & [name=Tetrade1] \\
	\psframebox{\parbox{4cm}{Rufe Funktion "`moddiv10"' mit dem Ergebnis der vorhergehenden ganzzahligen Division auf}} \\
	\psframebox{\parbox{4cm}{Modulus als oberes Nibble in Register 5 ablegen (dazu diesen um 4 Bits nach links shiften)}} \\
	~ & [name=Tetrade2] \\
	\psframebox{\parbox{4cm}{Rufe Funktion "`moddiv10"' mit dem Ergebnis der vorhergehenden ganzzahligen Division auf und speichere Modulus in Register 6}} \\
	~ & [name=Tetrade3] \\
	\psframebox{\parbox{4cm}{Register 5 auf Port 1 ausgeben}} \\
	\psframebox{\parbox{4cm}{Oberes Nibble von R0 als oberes Nibble in Port 6 ablegen}} \\
	\psframebox{\parbox{4cm}{Register 6 auf Port 2 ausgeben}} \\
	~ & [name=Loop2] & \makebox[5cm]{} & [name=Loop2a]
\end{psmatrix}
\ncline{->}{1,1}{2,1}
\ncline{->}{2,1}{4,1}
\ncline{->}{4,1}{6,1}
\ncline{->}{6,1}{8,1}
\ncline{->}{8,1}{9,1}
\ncline{->}{9,1}{11,1}
\ncline{->}{11,1}{13,1}
\ncline{->}{13,1}{14,1}
\ncline{->}{14,1}{15,1}
\ncangles[angleA=-90, armA=0.5cm, armB=2.5cm]{->}{15,1}{3,1}
\psbrace[ref=lC](Tetrade1)(Tetrade0){\parbox{3cm}{Berechung unterste BCD-Tetrade}}
\psbrace[ref=lC](Tetrade2)(Tetrade1){\parbox{3cm}{Berechung mittlere BCD-Tetrade}}
\psbrace[ref=lC](Tetrade3)(Tetrade2){\parbox{3cm}{Berechung obere BCD-Tetrade}}
\psbrace[ref=lC](Loop2)(Tetrade3){\parbox{3cm}{Ausgabe}}
\psbrace[ref=lC](Loop2a)(Init5a){\parbox{3cm}{Hauptschleife}}
\end{pspicture}
\end{center}
\caption{Bin"ar/BCD-Konvertierung f�r 7-Segment-Anzeige -- Detailbetrachtung Hauptschleife}
\label{abb:aufgabe33_2}
\end{figure}

\begin{figure}
\lstinputlisting[caption={Assembler-Programm f"ur Bin"ar/BCD-Konvertierung f�r 7-Segment-Anzeige -- Hauptschleife}, language=z8asm, label=lsg:aufgabe33_2, firstline=13, firstnumber=13, lastline=61]{aufgabe33.asm}
\end{figure}

\begin{table}
\begin{tabularx}{\textwidth}{|c|X|}
\hline
Register & Bedeutung \\
\hline
R0 & Von Port 0 (Ergebnis Analog-Digital-Umwandlung) ausgelesener Wert \\
5 & Ausgaberegister der unteren beiden BCD-Tetraden (nach Port 1) \\
6 & Hilfsregister f"ur Bitoperationen, Ausgaberegister obere BCD-Tetrade und LED-Balkenleiste (nach Port 2) \\
10 & "Ubergaberegister f"ur Parameter der Funktion "`moddiv10"' \\
11 & R"uckgabewert der Funktion "`moddiv10"': Modulus bez"uglich 10 \\
12 & R"uckgabewert der Funktion "`moddiv10"': Ergebnis der ganzzahligen Division bez"uglich 10 \\
\hline
\end{tabularx}
\caption{Registerbelegungstabelle f"ur die zweite Version (unter Ber"ucksichtigung von Schwingungen) des Programms zum Z"ahlen der Anzahl von Schalter"anderungen}
\label{table:aufgabe33}
\end{table}

Nun zur Funktion "`moddiv10"'. Diese nimmt in Register 10 einen Wert (ein vorzeichenloses Byte) entgegen und gibt in Register 11 den Modulus dieses Wertes bez"uglich 10 und in Register 12 den Wert der ganzzahligen Division von Register 10 durch 10 zur"uck. Der Grund, warum diese Funktion von uns programmiert, da der Z8-Microcontroller laut Anlage 11 der Versuchsanleitung keine Befehle f"ur ganzzahlige Division und Modulus bereitstellt.

Abbildung \label{abb:aufgabe33_3} beinhaltet das Flussdiagramm und Listing \label{lsg:aufgabe33_3} den zugeh"origen Quelltext.

\begin{figure}
\begin{center}
\begin{pspicture}(5,11)
%\psgrid
\begin{psmatrix}[rowsep=0.4,colsep=0.5]
	\psovalbox{Begin} \\
	\psframebox{\parbox{4cm}{Initialisiere Register 12 mit 0}}  \\
	~ & \\
	\psdiabox{\parbox{2.2cm}{Ist Register 10 $\leq$ 9}} \\
	\psframebox{\parbox{4cm}{Verringere den Wert in Register 10 um 10}} \\
	\psframebox{\parbox{4cm}{Inkrementiere Register 12}} \\
	~ & \\
	~ & \\
	\psframebox{\parbox{4cm}{Speichere Wert von Register 10 in Register 11}} \\
	\psovalbox{Return}
\end{psmatrix}
\ncline{->}{1,1}{2,1}
\ncline{->}{2,1}{4,1}
\ncline{->}{4,1}{5,1}\nbput{Nein}
\ncline{->}{5,1}{6,1}
\ncangles[angleA=-90, armA=0.5cm, armB=2.5cm]{->}{6,1}{3,1}
\ncangles[angleA=180, armA=0.5cm, armB=0.5cm, angleB=90]{->}{4,1}{9,1}\nbput{Ja}
\ncline{->}{9,1}{10,1}
\end{pspicture}
\end{center}
\caption{Funktion "`moddiv10"'}
\label{abb:aufgabe33_3}
\end{figure}

\begin{figure}
\lstinputlisting[caption={Assembler-Programm f"ur Bin"ar/BCD-Konvertierung f�r 7-Segment-Anzeige -- Funktion \texttt{moddiv10}}, language=z8asm, label=lsg:aufgabe33_3, firstline=64, firstnumber=64, lastline=78]{aufgabe33.asm}
\end{figure}

\clearpage
\subsection{Zeitmultiplexbetrieb der ADU-Kan�le}
\subsubsection{Aufgabe}

Erweitern Sie die Software von Aufgabe \ref{sec:aufg_3_2} derart, dass Sie im Zeitmultiplexverfahren beide Kanaleingangsspannungen des ADU digitalisieren. Jeder Kanal sollte dabei f�r ca. 4 s aktiviert werden (Software-Zeitschleife generieren).

\subsubsection{L"osung}

Da die Initialisierung der Ports und des Stacks, sowie die verwendete Funktion "`moddiv10"' gegen"uber der vorherigen Aufgabe nicht ver"andert wurden, wird auf ein erneutes Abdrucken dieser Programmteile verzichtet.

Wie in den Programmen davor, wird anschlie"send in die Hauptschleife "ubergegangen.

\begin{figure}
\lstinputlisting[caption={Assembler-Programm f"ur Zeitmultiplexbetrieb der ADU-Kan�le -- Hauptschleife}, language=z8asm, label=lsg:aufgabe34_1, firstline=13, lastline=22, firstnumber=13]{aufgabe34.asm}
\end{figure}

\begin{figure}
\lstinputlisting[caption={Assembler-Programm f"ur Zeitmultiplexbetrieb der ADU-Kan�le -- Routine \texttt{channel\_read\_and\_output}}, language=z8asm, label=lsg:aufgabe34_2, firstline=44, lastline=102, firstnumber=44]{aufgabe34.asm}
\end{figure}

\end{document}