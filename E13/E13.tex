\documentclass[10pt]{scrbook}
\usepackage[latin1]{inputenc}
\usepackage{bibgerm}
\usepackage[english, ngerman]{babel}
\usepackage[babel]{csquotes}
\usepackage{cite}
\usepackage{pst-plot}
\usepackage{tabularx}
\author{Denis Dietze\footnote{denis.dietze@st.ovgu.de} \and Stephan Ebeling\footnote{red\_crow@t-online.de} \and Wolfgang Keller\footnote{wolfgang.keller@student.uni-magdeburg.de} \and Nico Linke\footnote{nicolinke@googlemail.com} \and Thomas Schulte\footnote{thomas.schulte.md@googlemail.com}}
\title{Protokoll Versuch IT1/E13}
\subtitle{Schnittstellen}

\begin{document}
\maketitle
\tableofcontents
\chapter{Vorbereitungsaufgaben}

\section{Baugruppen eines Rechners}

\subsection{Frage} Aus welchen Baugruppen besteht ein Rechner �hnlich der von-Neumann-Architektur,
wie sind sie verbunden, und wie erfolgt die Kommunikation mit der Peripherie?

\subsection{Antwort}
Laut \cite[S.~41ff]{Schiffmann} besteht ein Von-Neumann-Rechner aus einem Schaltwerk, Speicher und Eingabe/Ausgabe. Das Schaltwerk besteht hierbei aus Rechenwerk und Leitwerk.

Vom Leitwerk geht ein Adress- und Steuerbus zu den drei anderen Baugruppen. Der Datenbus besteht bidirektional zwischen Speicher zum Rechenwerk, sowie zwischen Rechenwerk und Ein-/Ausgabe. Au�erdem besteht eine Datenbusverbindung vom Speicher zum Leitwerk.


\section{Aufgabe von Schnittstellen}

\subsection{Frage} Welche Aufgabe haben Schnittstellen im Rechner?

\subsection{Antwort}
Schnittstellen dienen zum Anschluss von Peripherieger"aten.

\section{Datenaustausch �ber eine parallele Schnittstelle mit
Handshaking}

\subsection{Frage} Wie funktioniert ein Datenaustausch �ber eine parallele Schnittstelle mit
Handshaking?

\subsection{Antwort}
Die konkreten Details des Austausch h"angen davon, in welchem Modus der Parallelport betrieben wird. Daher unterscheiden wir drei F"alle:

\begin{itemize}
\item 8255 im Modus 1 (im Modus 0 findet kein Handshaking statt)
\item 8255 im Modus 2
\item Centronics-Schnittstelle
\end{itemize}

\subsubsection{8255 im Modus 1}

Je drei Leitungen des Port PC werden laut \cite[S.~441]{Baehring} den Ports PA und PB als Handshake-Leitungen (\verb|STB|, \verb|ACK| und \verb|INT| zugeordnet. 

\paragraph{Eingabe} Die "Ubertragung beginnt damit, dass das Ger"at ein Datum in \verb|PA0|-\verb|PA7| bzw. \verb|PB0|-\verb|PB7| gelegt wird. Anschlie�end wird ein kurzes \verb|STB|-Signal erzeugt. Das Ausgangssignal \verb|IBF| zeigt dem Ger"at an, dass das Eingaberegister \verb|IB| noch gef"ullt ist.

\ldots

\subsubsection{Centronics-Schnittstelle}

Laut \cite[S.~447-448]{Baehring} beginnt die "Ubertragung "uber die Centronics-Schnittstelle damit, dass der Prozessor ein Ausgabedatum auf auf die Ausg"ange \verb|DATA1|-\verb|DATA8| der Schnittstelle schaltet. Fr"uhestens $0,5 \mu{}s$ sp"ater wird das \verb|STROBE|-Signal aktiviert, um das Vorliegen eines g"ultigen Datums anzuzeigen.

Anschlie�end kann das Ger"at die Daten entweder annehmen oder durch Aktivieren eines \verb|BUSY|-Signals anzeigen, dass es besch"aftigt ist. Wenn es bereit ist die Daten anzunehmen, deaktiviert es das \verb|BUSY|-Signal und quittiert die Datenannahme durch ein mindestens $0,5 \mu{}s$ langes \verb|ACKNLG|-Signal.

In \cite[S.~9]{PreussMusa} wird weiter zwischen einem \emph{Dreidraht-Handshake-Protokoll}, was dem soeben Beschriebenen entspricht und einem \emph{Zweidraht-Handshake-Protokoll} unterschieden.

In letzterem bleibt beim Anlegen der Daten auf den Datenleitungen die \verb|STROBE|-Leitung so lange aktiviert, bis der Rechner ein Signal auf der \verb|ACKNOWLEDGE|-Leitung empf"angt. Dieses setzt das \verb|STROBE|-Signal zur"uck und das Peripherieger"at kann neue Daten empfangen.

\section{Funktionsweise einer seriellen Daten�bertragung}

\subsection{Frage} Wie funktioniert eine serielle Daten�bertragung?

\subsection{Antwort}

Laut \cite[S.~455ff]{Baehring} schreibt der Sender ein auszugebendes Datum in das Sende-Datenregister \verb|TDR|. Von der Bausteinsteuerung wird es automatisch in das Sende-Schieberegister \verb|TSR| "ubertragen. Von der Sende-Synchronisationsschaltung wird das Startbit die Daten-Ausgabeleitung \verb|TxD| gegeben. Anschlie�end wird durch eine Anzahl Impulse als Sendetakt das Datum Bit f"ur Bit "ubertragen -- mit jedem Takt wird ein Bit aus dem Register auf \verb|TxD| geschoben.

Parallel dazu wird die Parit"at ausgerechnet und nach dem letzten Bit "ubertragen. Nach Abschluss der "Ubertragung erzeugt die Synchronisationsschaltung ein oder mehrere Stoppbits.

Wenn mehrere Daten "ubertragen werden sollen, so wird durch die Ausgangsleitung \verb|TxRDY| dem Prozessor mitgeteilt, dass der Inhalt von \verb|TDR| nach \verb|TSR| "ubertragen wurde.

Der Empfang funktioniert analog: auf eine negative Flanke von \verb|RxD| wird die Empfangs-Synchronisationsschaltung aktiviert. Die folgenden Bits werden in das Empfangs-Schiebe\-register \verb|RSR| eingelesen. Ein empfangenes Datum wird in \verb|RDR| abgelegt. "Uber den Ausgang \verb|RxRDY| teilt der Baustein dem Prozessor mit, wenn in \verb|RDR| ein Datum zur Abholung bereit liegt.

Parallel wird ein Parit"atsbit berechnet. Wenn dieses nicht mit dem "ubertragenen Parit"atsbit "ubereinstimmt, wird im Statusregister ein Parit"atsfehler festgehalten.

Zuletzt wird "uberpr"uft, ob die geforderte Anzahl Stoppbits "ubertragen wurde. Falls dies nicht der Fall ist, wird im Statusregister ein Bit gesetzt (Framing Error). Au�erdem wird ebenfalls ein Statusbit gesetzt, wenn ein neues Datum empfangen wird, der Prozessor aber noch nicht das alte abgeholt hat (Overrun Error).

\section{Das Prinzip der Stromschleife}

\subsection{Frage} Erl�utern Sie das Prinzip der Stromschleife! Was ist der Vorteil gegen�ber einer
spannungsbasierten �bertragung?

\subsection{Antwort}

Die Stromschleife besteht aus einer aktiven (meist PC) und passiven (meist Peripherieger"at) Seite. Wie im Bild in Abschnitt 2.1.2 der Versuchsanleitung zu sehen, wird ein Strom von $20~mA$ "uber diese gesendet, wenn eine logische 1 "ubertragen wird und kein Strom, wenn eine logische 0 "ubertragen wird. Auf der passiven Seite wird durch einen Optokoppler bzw. Relais das Signal ausgewertet.

Der Vorteil gegen"uber einer spannungsbasierten "Ubertragung besteht darin, dass nicht beachtet werden muss, ob die beiden Seite unterschiedliche elektrische Potentiale besitzen, da Sender und Empf"anger voneinander isoliert sind.

\section{Merkmale des PPI 8255}

\subsection{Frage} Nennen Sie wesentliche Merkmale des PPI 8255! Wie funktionieren die
Betriebsarten?

\subsection{Antwort} 3 Betriebsmodi (vgl. \cite[S.~440-447]{Baehring})
\begin{itemize}
\item Modus 0
\item Modus 1
\item Modus 2
\end{itemize}

\subsubsection{Modus 0}

\begin{itemize}
\item Port \verb|PC| in zwei Teilports $\verb|PC|_{\verb|H|}$ und $\verb|PC|_{\verb|L|}$ aufgeteilt
\item Jeder Teilport als paralleler Ein-/Ausgabeport mit immer selben "Ubertragungsrichtung (wird f"ur jeden der \verb|PA|, \verb|PB| $\verb|PC|_{\verb|H|}$ und $\verb|PC|_{\verb|L|}$ im Steuerwort individuell festgelegt)
\end{itemize}

\subsubsection{Modus 1}

\begin{itemize}
\item Je drei Leitungen von \verb|PC| werden \verb|PA| und \verb|PB| als Handshake-Leitungen zugeordnet
\item Restliche beiden Leitungen von \verb|PC| werden \verb|PA| zugeordnet
\end{itemize}

\subsubsection{Modus 2}


\begin{itemize}
\item Von Port \verb|PC| werden drei Leitungen \verb|PB| zugeordnet
\item \verb|PB| entweder Ein- oder Ausgangsport
\item Die drei Leitungen von \verb|PC| wahlweise f"ur E/A oder Handshake
\item Restliche f"unf Leitungen von \verb|PC| f"ur Handshake von \verb|PA| (erlaubt bidirektionale "Ubertragung im Halbduplex-Modus)
\end{itemize}

\section{Interruptverarbeitung im PC}

\subsection{Frage} Wie ist die Interruptverarbeitung im PC organisiert? Wie sind CPU,
Interruptcontroller und -quellen miteinander verschaltet? Wie funktioniert
Vektorinterrupt?

\subsection{Antwort} \ldots (Antwort bekannt -- zu faul jetzt aufzuschreiben)

\section{Hardware- vs. Software-Interrupts}

\subsection{Frage}
Was unterscheidet Hardware- von Software-Interrupts, was haben sie gemeinsam?

\subsection{Antwort}
Hardware-Interrupts werden durch ein Peripherie-Ger"at ausgel"ost, w"ahrend Software-Interrupts durch einen \verb|INT|-Befehl ausgel"ost werden.

In beiden F"allen wird laut \cite[S.~107]{althaus}  der Wert des Stack Pointers \verb|SP| und des Flag-Registers auf dem Stack gesichert und der dem Interrupt zugeh"orige Wert aus der Interrupt-Einsprungtabelle in den Program Counter \verb|PC| geladen.

Somit stellen Software-Interrupts eine Art Funktionsaufruf dar, bei dem zus"atzlich das Flag-Register gesichert und wiederhergestellt wird, w"ahrend Hardware-Interrupts eine asynchrone Unterbrechung des Programmablaufs darstellen.

\section{M"oglichkeiten der Schnittstellenkonfigurierung}

\subsection{Frage} Informieren Sie sich �ber M"oglichkeiten der Schnittstellenkonfigurierung. Welche
Rolle spielen dabei Softwareinterrupts?

\subsection{Antwort}

(vgl. Versuchsbeschreibung, Abschnitt 2.4):
\begin{itemize}
\item Direkter Zugriff im E/A-Adressraum
\item Verwendung von BIOS-Interrupts
\item Verwendung von DOS-Interrupts
\end{itemize}

Letztere beide M"ochlichkeiten sind Softwareinterrupts.

\section{Befehle zum Zugriff auf Ein-/Ausgabe-Baugruppen}

\subsection{Frage} Wiederholen Sie die Syntax des 80x86-Assemblers (siehe Anlage 12 der Versuchsanleitung). Mit welchen
Befehlen wird auf die Ein-/Ausgabe-Baugruppen zugegriffen?

\subsection{Antwort} Die Befehle zum Zugriff auf Ports sind \verb|IN| zum Einlesen und \verb|OUT| zum Ausgeben eines Wertes aus einem Port. Au�erdem sind viele Baugruppen unter DOS "uber Interrupts ansteuerbar, wie in Anlagen 3-5 der Versuchsanleitung bez"uglich Interrupt 14h, 15h bzw. 21h f"ur die serielle bzw. parallele Schnittstelle. In diesem Fall erfolgt der Zugriff durch Belegen der Prozessorregister mit den gew"unschten Werten gefolgt von einem \verb|INT|-Befehl.

\chapter{Aufgaben und Auswertung}

\section{Verwendete Notation}

Wir haben uns daf"ur entschieden, die Ein- und Ausgaben von \verb|debug| m"oglicht exakt so wiederzugeben, wie diese auf dem Bildschirm erscheinen. Da im Allgemeinen klar ist, was Ausgabe von \verb|debug| und Eingabe darstellt, ist dies kein Problem.

\section{Aufgabe 1}

\subsection{Aufgabenstellung} �berpr�fen Sie die Kabelverbindungen, wie in obiger Abb. gezeigt, zwischen Steuerpult
und PC sowie den Peripherieger�ten, wie A/N-Display oder Drucker. Beachten Sie die
am Versuchsplatz ausliegende Mappe mit zus�tzlichen Informationen. Booten Sie den PC
und kontrollieren Sie die Anmeldung der verf�gbaren Schnittstellen im "Setup", und
verschaffen Sie sich einen �berblick �ber die Rechnerkonfiguration mit dem
Hilfsprogramm \verb|si| bzw. \verb|debug|.

\subsection{Durchf"uhrung und Auswertung}
\label{sec:aufg1_durchf}

\verb|si| liefert folgende f"ur die L"osung der Aufgabe relevanten Informationen:

\paragraph{Schnittstellen}
\begin{itemize}
\item Serielle Ports: 2
\item Parallele Ports: 1
\end{itemize}

\paragraph{Hardware-Interrupts}

\begin{tabular}[t]{|l|l|l|}
\hline
Nummer & Adresse & Name \\
\hline
IRQ 03 & \verb|0DCA:006F| & \verb|COM2| \\
IRQ 04 & \verb|0DCA:0087| & \verb|COM1| \\
IRQ 07 & \verb|0070:06F4| & \verb|LPT1| \\
\hline
\end{tabular}

\paragraph{Software-Interrupts}

\begin{tabular}[t]{|l|l|l|}
\hline
Nummer & Name & Adresse  \\
\hline
\verb|0B| & IRQ 03 & \verb|0DCA:006F| \\
\verb|0C| & IRQ 04 & \verb|0DCA:0087| \\
\verb|0F| & IRQ 7 - Drucker & \verb|0070:06F4| \\
\verb|17| & Drucker & \verb|0070:06F4| \\ 
\hline
\end{tabular}

\section{Aufgabe 1a}

\subsection{Aufgabenstellung} Ab Adresse Segment:Offset=0:410h speichert das Betriebssystem das
Systemkonfigurations-Wort (16 Bit) ab. Bits 15 und 14 enthalten die Anzahl der
parallelen und Bits 9 und 10 die Anzahl der seriellen Ports. Ab Adresse 0:400h stehen die
Basisadressen von COM1 und COM2 und ab Adresse 0:408h die Basisadressen von
LPT1-3 (jeweils 16 Bit). Zeigen Sie die Werte mit <d> an. Notieren Sie sich die Werte.

\subsection{Durchf"uhrung} alle folgenden Ausgaben notieren:

\begin{verbatim}
-d0:410 l 2
0000:0410  21 44
\end{verbatim}

Man beachte, dass die Ausgabe in Little-Endian-Darstellung ist. Daher hat das Systemkonfigurations-Wort den Wert \verb|55 21|. Bit 15 und 14 haben den Wert \verb|01| -- wir haben also einen parallelen Port. Der Wert von Bits 10 und 9 ist \verb|10| -- also zwei serielle Ports.

Als n"chstes ermitteln wir die Basisadressen der beiden seriellen Ports:

\begin{verbatim}
-d0:400 l 4
0000:0400  F8 03 F8 02
\end{verbatim}

Die Basisadresse von COM1 (Little-Endian-Darstellung in der Ausgabe von Debug beachten) ist also \verb|03F8| und die von COM2 ist \verb|02F8|.

Zuletzt ermitteln wir die Basisadresse von LPT1-3:

\begin{verbatim}
-d0:408 l 6
0000:0400                          BC 03 00 00 00 00
\end{verbatim}

Die Basisadresse von LPT1 ist also \verb|03BC|, w"ahrend f"ur LPT2 und LPT3, da diese nicht existieren, vom System die Adresse \verb|0000| abgespeichert wird.

\section{Aufgabe 1b}

\paragraph{Aufgabenstellung} Auf welcher Speicheradresse befindet sich der Interruptvektor von IRQ 7 und welchen
Wert hat er? Vergleichen Sie den Wert, wie er mit 'si' angezeigt wird, mit der Anzeige in
'debug'. Lassen Sie sich die Interrupt-Serviceroutine als Assemblercode anzeigen
(Befehl \textless u\textgreater ).

\paragraph{Durchf"uhrung} Laut Tabelle 9 der Versuchsanleitung bzw. der zweiten Tabelle aus \ref{sec:aufg1_durchf} hat IRQ 7 die Nummer INT 0F. Da die Interruptvektortabelle an an der Adresse 0:0 beginnt und jeder Eintrag vier Bytes lang ist, liegt der Interruptvektor von IRQ 7 an der Speicheradresse \verb|3C|.

Um den Wert anzuzeigen, benutzen wir also

\begin{verbatim}
-d0:3C l 4
0000:0030                                                  F4 06 70 00
\end{verbatim}

Der Interruptvektor von IRQ 7 hat also die Adresse \verb|0070:06F4|. Wie man im Vergleich mit den Tabellen aus Abschnitt \ref{sec:aufg1_durchf} sehen kann, ist die Darstellung von \verb|debug| in Little-Endian-Darstellung, w"ahrend \verb|si| die Speicheradresse in Big-Endian-Darstellung anzeigt.

Um den Assemblercode anzuzeigen, benutzen wir
\begin{verbatim}
-U 0070:06F4
0070:06F4 CF            IRET
\end{verbatim}
(in der Ausgabe von \verb|debug| folgen zahlreiche weitere Zeilen). Wie man sieht, besteht die Interrupt Service Routine ausschlie�lich aus einem \verb|IRET|-Befehl, welcher f"ur einen sofortigen R"ucksprung aus der Interrupt Service Routine sorgt. Daher sind die folgenden Zeilen der Ausgabe von \verb|debug| f"ur die gestellte Aufgabenstellung unerheblich und wurden daher weggelassen.

\section{Aufgabe 2}

\paragraph{Aufgabenstellung} Initialisieren Sie den PPI 8255 der E/A-Karte PIO-12 im Mode 1 (Portadressen siehe
Anlagen 6 und 8) f�r Byteeingabe Port A, Byteausgabe Port B mit Handshaking-
Protokoll an Port C. Ermitteln Sie die notwendigen Steuerworte anhand Anlage 7 und
geben Sie diese mit Hilfe des DOS-Debuggers auf die Steuerwortaderesse des PPI aus
(Befehls�bersicht in Anlage 2). Simulieren Sie einen asynchronen Datenaustausch mit
Hilfe der Leuchtdioden und Schalter des Steuerpults, indem Sie mit dem Debugger Einund
Ausgaben durchf�hren. Stellen Sie das Signalspiel auf dem Digitaloszilloskop dar
(Digitalmodus, Ablenkung ca. 1-2 s).


\paragraph{Durchf"uhrung}
\label{sec:aufg2_durchf}
Um auf Mode 1 mit den gew"unschten Eigenschaften zu wechseln, m"ussen wir das Steuerwort von LPT1 auf \verb|10110100| (\verb|b4|) setzen. Der Wert setzt sich folgenderma�en zusammen (f�r die Bedeutung der Bits vgl. oberen Teil der von Anlagen 7 der Versuchsanleitung):
\begin{itemize}
\item Bit 7: auf 1 setzen, da wir den Modus setzen wollen
\item Bit 6 und 5: auf 01 setzen, da wir Mode 1 f"ur Port A benutzen wollen
\item Bit 4: auf 1 setzen, da Port A als Input dienen soll
\item Bit 3: da Port A f"ur Mode 1 initialisiert wird, kann dieses Bit ignoriert werden -- wir setzen es auf 0
\item Bit 2: auf 1 setzen, da wir Mode 1 f"ur Port B benutzen wollen
\item Bit 1: auf 0 setzen, da Port B als Output dienen soll
\item Bit 0: da Port A f"ur Mode 1 initialisiert wird, kann dieses Bit ignoriert werden -- wir setzen es auf 0
\end{itemize}

Da das Steuerwort von LPT1 die Adresse \verb|303h| hat (vgl. Tabellen in den Anlagen 6 und 8 der Versuchsanleitung), ergibt sich somit:

\begin{verbatim}
-O 303 b4
\end{verbatim}

Als ersten Schritt simulierten wir eine "Ubertragung vom Peripherieger"at zum Rechner. Dazu stellen wir auf dem PPI-12 auf Port A den Wert \verb|A2| ein.

Da $\textnormal{PC}_\textnormal{4}$ ein negiertes \verb|STB|-Signal darstellt, deaktivieren wir dieses, um \verb|STB| zu aktivieren. Wir sehen an den LEDs des PPI-12, dass $\textnormal{IBF}_\textnormal{A}$ aktiviert wird. Da das \verb|STB|-Signal nur eine kurze Flanke darstellt, deaktivieren wir es anschlie�end wieder.

Nun tippen wir in \verb|debug| ein:

\begin{verbatim}
-I 300
A2
\end{verbatim}

\verb|300| ist -- wie man in den Tabellen in den Anlagen 6 und 8 der Versuchsanleitung sehen kann -- die E/A-Adresse von Port A.

Wir beobachten, dass au�er, dass \verb|debug| das zu "ubertragende Byte anzeigt, au�erdem PC5 ($\textnormal{IBF}_\textnormal{A}$) sich ausschaltet.

Auf dem Oszillographen sieht dies folgenderma�en aus:

\psset{xunit=0.8cm,yunit=0.8cm}
\begin{center}
\begin{pspicture}(0,0)(15,10)
\readdata{\stb}{stb.dat}
\readdata{\ibfa}{ibfa.dat}
\dataplot[plotstyle=line,linecolor=blue]{\stb}
\dataplot[plotstyle=line,linecolor=blue]{\ibfa}
\psline{->}(0,0)(12,0)
\psline[linestyle=dashed]{-}(0,5)(12,5)
\psline{->}(0,0)(0,10)
\rput[l](12.2,8){$\overline{\texttt{STB}}$ ($\texttt{PC}_\texttt{4}$)}
\psline{-}(4,8.2)(4,8.9)
\rput[r](4,9.2){$\overline{\texttt{STB}}$ deaktivieren}
\psline{-}(5,8.2)(5,8.9)
\rput[l](5,9.2){$\overline{\texttt{STB}}$ aktivieren}
\rput[l](12.2,2.5){$\texttt{IBF}_\texttt{A}$ ($\texttt{PC}_\texttt{5}$)}
\psline{-}(7.5, 2.4)(7.5, 1.6)
\rput[c](7.5,1.3){In \texttt{debug} "`\texttt{-I 300}"' eintippen}
\end{pspicture}
\end{center}

Zur Erkl"arung: wenn wir \verb|STB| auf $\textnormal{PC}_\textnormal{4}$ aktivieren, so wird zur"uckgegeben, dass der Input Buffer von Port A voll ist ($\textnormal{IBF}_\textnormal{A}$ auf $\textnormal{PC}_\textnormal{5}$). Sobald dieses ausgelesen wurde, wird dieses Signal zur"uckgenommen und es kann ein weiteres Byte "ubertragen werden.

Nun kommen wir zur Ausgabe auf Port B. Hierzu tippen wir in \verb|debug| ein:
\begin{verbatim}
-O 301 32
\end{verbatim}
Hierbei stellt 301 die E/A-Adresse von Port B dar (vgl. Tabellen in den Anlagen 6 und 8 der Versuchsanleitung).

Nachdem dieser Befehl abgesetzt wurde, schalten wir das negierte ACK ($\textnormal{PC}_\textnormal{2}$) aus und wieder ein.

\psset{xunit=0.8cm,yunit=0.8cm}
\begin{center}
\begin{pspicture}(0,0)(15,10)
\readdata{\obfb}{obfb.dat}
\readdata{\ack}{ack.dat}
\dataplot[plotstyle=line,linecolor=blue]{\obfb}
\dataplot[plotstyle=line,linecolor=blue]{\ack}
\psline{->}(0,0)(12,0)
\psline[linestyle=dashed]{-}(0,5)(12,5)
\psline{->}(0,0)(0,10)
\rput[l](12.2,8){$\overline{\texttt{OBF}_\texttt{B}}$ ($\texttt{PC}_\texttt{1}$)}
\psline{-}(3,8.2)(3,8.9)
\rput[l](3,9.2){In \texttt{debug} "`\texttt{-O 301 32}"' eintippen}
\rput[l](12.2,2.5){$\overline{\texttt{ACK}}$ ($\texttt{PC}_\texttt{2}$)}
\psline{-}(5,1.4)(5,0.9)
\rput[r](5,0.7){$\overline{\texttt{ACK}}$ deaktivieren}
\psline{-}(8,1.4)(8,0.9)
\rput[l](8,0.7){$\overline{\texttt{ACK}}$ aktivieren}
\end{pspicture}
\end{center}

Auch hier eine Erkl"arung: nachdem die Daten in den Output Buffer geschoben wurden und somit der Output Buffer voll ist {$\textnormal{OBF}_\textnormal{B}$), wird, sobald die Daten empfangen wurden, ein ACK gesendet. Dieses wird anschlie�end wieder zur"uckgenommen, so dass weitere Daten gesendet werden k"onnen.

\section{Aufgabe 3}

\subsection{Durchf"uhrung}

Als ersten Schritt initialisieren wir das Steuerwort (der Grund, warum das Steuerwort auf \verb|b4| zu setzen ist, ist Abschnitt \ref{sec:aufg2_durchf} zu entnehmen).

\begin{verbatim}
-O 303 b4
\end{verbatim}

Wie Abschnitt \label{sec:aufg1_durchf} bzw. der Tabelle in Anlage 9 der Versuchsanleitung zu entnehmen ist, hat der IRQ 7 von \verb|LPT1| den Eintrag mit Nummer \verb|0F| in der Interruptvektortabelle. Da jeder Eintrag 4 Bytes hat (16 Bit Offset+16 Bit Segment), liegt der zu IRQ 7 korrespondierende Eintrag in der Interruptvektortabelle an der Adresse \verb|4*OF=3c|, was in der segmentierten Notation der Adresse \verb|0:3c| entspricht.

Wir wollen die Interrupt Service Routine an der Adresse \verb|500:0| ablegen. Somit geben wir in \verb|debug| ein:

\begin{verbatim}
-e0:3c 00 00 00 05
\end{verbatim}

Als folgenden Schritt assemblieren wir die Interrupt Service Routine an genannter Adresse \verb|500:0|:

\begin{verbatim}
-a 500:0
0500:0000 push ax
0500:0001 push dx
0500:0002 mov dx, 300
0500:0005 in al, dx
0500:0006 mov dx, 301
0500:0009 out dx, al
0500:000A mov al, 20 ; Ruecksetzbefehl des PIC (EOI)
0500:000C mov dx, 20 ; Portadresse PIC
0500:000F out dx, al
0500:0010 pop dx
0500:0011 pop ax
0500:0012 iret
0500:0013
\end{verbatim}

Nun lesen wir den Wert des Interruptmaskenregisters ein
\begin{verbatim}
-i21
98
\end{verbatim}
(also bin"ar \verb|10011000|) und setzen Bit 7 auf 0, um die Maskierung von Interrupt 7 zu entfernen
\begin{verbatim}
-o 21 18
\end{verbatim}

Als n�chstes muss das ei-Flag der CPU gesetzt werden:
\begin{verbatim}
-rf
NV UP EI PL NZ NA PO NC  -ei
\end{verbatim}
Wie man sieht, war dieses bereits aktiviert, weswegen die Eingabe keine "Anderung bewirkt.

Wir wollen einstellen, dass durch ein Handshaking der Interrupt aktiviert werden soll. Dazu betrachten wir die letzte Grafik von Anlage 7 der Versuchsanleitung: diese sagt, dass f"ur den in die E/A-Adresse \verb|303| einzutragenden Wert
\begin{itemize}
\item Bit 7 auf 0 zu setzen ist
\item Bits 6-4 don't care sind (wir setzen sie auf 0
\item Bits 3-1 auf 100 zu setzen sind, da PC4 im Input Mode INTEA steuert, somit m"ussen wir in der Spalte 4 schauen
\item Bit 0 auf 1 zu setzen ist
\end{itemize}

Somit m"ussen wir \verb|09| in die E/A-Adresse 303 schreiben:

\begin{verbatim}
-o 303 09
\end{verbatim}

Damit die geforderte Aufgabenstellung umgesetzt wird, ist noch $\texttt{PC}_\texttt{3}$ mit $\texttt{INT}_\texttt{IN}$ zu verbinden und der Schalter \verb|EN| anzuschalten.

Nun k"onnen wir tats"achlich das Experiment durchf"uhren: dazu stellen wir auf Port A etwas ein (in unserem Fall \verb|12|), schalten $\texttt{PC}_\texttt{4}$ (negiertes \verb|STB|) aus und wieder ein. Nun sehen wir, wie der eingestellte Wert auf Port B erscheint.

\section{Aufgabe 5}

\paragraph{Aufgabe} Geben Sie mit Hilfe von BIOS-Funktionen Daten vom PC �ber die serielle Schnittstelle
auf das A/N-Display aus. Beachten Sie die notwendige Initialisierung der seriellen
Schnittstelle (2400 Baud, 8 Bit Daten). Stellen Sie das Datentelegramm auf dem
Oszilloskop dar (Digitalmodus, Ablenkung ca. 1 ms) und protokollieren Sie es.

\paragraph{Durchf"uhrung}

\begin{verbatim}
-A600:0
0600:0000 mov ah, 0   ; Initialisiere Schnittstelle
0600:0002 mov dx, 1   ; 0: COM1, 1: COM2
0600:0005 mov al, bf
0600:0007 int 14
0600:0009 mov ah, 1   ; Sende Zeichen
0600:000B mov al, 48
0600:000D int 14
0600:000F mov ah, 1   ; Sende Zeichen
0600:0011 mov al, 69
0600:0013 int 14
0600:0015 mov ah, 1   ; Sende Zeichen
0600:0017 mov al, 21
0600:0019 int 14
0600:001B

-G =600:0 600:1B
\end{verbatim}

\psset{xunit=0.6cm,yunit=0.6cm}
\begin{center}
\begin{pspicture}(0,0)(20,7)
\readdata{\serielle}{serielle_schnittstelle.dat}
\dataplot[plotstyle=line,linecolor=blue]{\serielle}
\psline{->}(0,0)(20,0)
\psline[linestyle=dashed]{-}(0,3)(20,3)
\psline{->}(0,0)(0,7)
\end{pspicture}
\end{center}

\bibliography{E13}{}
\bibliographystyle{geralpha}

\end{document}