\documentclass[10pt]{scrartcl}
\usepackage[latin1]{inputenc}
\usepackage[english, ngerman]{babel}
\usepackage[babel]{csquotes}
\usepackage{amsmath} % for eqref
\usepackage{tabularx}
\usepackage{listings} % for listings - of course ;-)
\usepackage{graphicx}
\usepackage[usenames,dvipsnames]{color}
\author{Denis Dietze \and Wolfgang Keller \and Nico Linke \and Thomas Schulte}

\title{Protokoll Versuch IT2}
\subtitle{Microcontroller Z8}
\begin{document}
\maketitle

\lstdefinelanguage{z8asm}
{
morekeywords={ORG,LD,TITLE,CALL,NOP,RET,END,JP},
morecomment=[l]{;}
}

\lstset{
numbers=left,                   % where to put the line-numbers,
commentstyle=\color{OliveGreen},
showtabs=true,                 % show tabs within strings adding particular underscores
frame=single,	                % adds a frame around the code
tabsize=2	                % sets default tabsize to 2 spaces
}

\section{Port-1-Komplex}

\subsection{Konfiguration von Port 1 zur Speichererweiterung durch 256 Byte externen RAM}
\label{sec:1a}

\subsubsection{Aufgabe}
Konfigurieren Sie Port 1 zur Speichererweiterung durch 256 Byte externen RAM.

\subsubsection{L"osung}

Wir m"ussen in das Register \verb|P01M| den Bin"arwert \verb|XX0100XX| schreiben, wobei X f"ur einen beliebiges Bit steht. Wir haben uns f"ur den Wert \verb|10h| entschieden.

Der Befehl lautet somit: \verb|LD P01M, #%10|.

\subsection{Adressen zum Zugriff auf den externen RAM}
\label{sec:1b}

\subsubsection{Aufgabe}
�ber welche Adressen kann auf den externen RAM zugegriffen werden?

\subsubsection{L"osung}
Die ersten 4096 (\verb|1000h|) Adressen sind durch das interne ROM belegt. Alle dar�ber hinaus gehenden Adressen der mit den 13 Adressbits A0-A12 darstellbaren Adressen (also von \verb|1000h| bis \verb|3000h|) geh�ren somit \emph{theoretisch} zum externen RAM.

Man kann in der Tat �ber all diese Adressen auf den externen RAM zugreifen. Jedoch tritt in der Praxis eine Besonderheit auf: wie man in ersten Grafik Anlage 1 der Versuchsbeschreibung sehen kann, sind die Adressbits A8 bis A12 mit Masse verbunden - d. h. im RAM sind nur 256 (\verb|100h|) verschiedene Adressen adressierbar. Die obersten 5 Adressbits (A8-A12) werden ignoriert. Da somit nur 256 Bytes an RAM physikalisch adressierbar sind, wird der Inhalt dieser 256 Bytes in den restlichen mit A0 bis A12 adressierbaren Adressen des externen RAMs zyklisch wiederholt.

\subsection{Befehle zum Zugriff auf den RAM}
\label{sec:1c}

\subsubsection{Aufgabe}

Mit welchen Befehlen kann auf das RAM zugegriffen werden?

\subsubsection{L"osung}

TODO: Nico und Thomas: Inhalt einf"ugen

\subsection{Nutzung eines externen Stacks}
\label{sec:1d}

\subsubsection{Aufgabe}
Programmieren Sie einen externen Stack und legen Sie seine Adresse auf das Ende des unter \ref{sec:1a} konfigurierten externen RAMs. Testen Sie im Schrittbetrieb mittels eines \verb|CALL|-Befehls zur Adresse \verb|500h| und eines \verb|RETURN|-Befehls von dort die korrekte Funktion des Stacks.

Wo liegt exakt die R�cksprungadresse im Stack?

\subsubsection{L"osung}

Gem"a"s den Instruktionen w"ahrend des Praktikums wurde die Aufgabenstellung derart erweitert, dass vor dem \textbf{RETURN}-Befehl ein weiterer Sprung auf die Adresse \verb|600h| mit direkt folgendem R�cksprung erfolgen soll.

\lstinputlisting[caption={Nutzung eines externen Stacks},language=z8asm]{aufgabe31d.asm}

Die R"uckkehradresse liegt dann in der Adresse \verb|0x10FE| auf dem Stack, da der Program Counter 2 Bytes ben�tigt und entsprechend durch den \verb|CALL|-Befehl wie in Anlage 9 beschrieben der Stack Pointer \verb|SP| um 2 erniedrigt wird und in dieser Adresse der Program Counter \verb|PC| abgespeichert wird.

Wie man in den Zeile 6 und 7 sehen kann, werden SPH bzw. SPL und  (High bzw. Low Byte des Stack Pointers) auf die Adressen \verb|11h| bzw. \verb|00h| gesetzt. Nach den Vorbetrachtungen in Abschnitt \label{sec:1b} stellt dies das Ende des externen unter \label{sec:1a} konfigurierten externen RAMs dar.

Wenn wir uns im Einzelschritt-Betrieb in Zeile 9 befinden, so ist offensichtlich noch kein Wert auf den Stack geschrieben worden. Der Program Counter PC hat den Wert \verb|0015h|.

Wenn wir einen Schritt weitergehen (den \verb|CALL|-Befehl also ausf�hren), so wird in der Speicheradresse \verb|10FFh| der Wert \verb|00h| und in der Speicheradresse \verb|10FEh| der Wert \verb|18h| abgespeichert. Dies stellt ein Pushen der Adresse des folgenden und Befehls (Zeile 10) in Little-Endian-Darstellung auf den Stack dar. Zudem wurde der Wert des aus SPL und SPH bestehenden Stack Pointers um 2 erniedrigt.

Die Werte sind am Anfang von Zeile 15 also folgenderma�en:
\\

\begin{tabular}{|lr|}
\hline
Register/Speicheradresse & Wert \\
\hline
PC & \verb|500h| \\
SPL & \verb|FE| \\
SPH & \verb|10| \\
\verb|10FEh| & \verb|18h| \\
\verb|10FFh| & \verb|00h| \\
\hline
\end{tabular}
\\

Wenn wir nun einen Schritt weitergehen und Zeile 15 ausf�hren, so wird der Wert \verb|503h| (der \verb|CALL|-Befehl ist 3 Bytes lang, die zur Adresse \verb|0500h| addiert werden) auf den Stack gepusht und der Program Counter auf den Wert \verb|600h| gesetzt.

Somit sind nach Ausf�hrung dieser Zeile in den Registern bzw. Speicheradressen folgende Werte abgelegt:
\\

\begin{tabular}{|lr|}
\hline
Register/Speicheradresse & Wert \\
\hline
PC & \verb|600h| \\
SPL & \verb|FC| \\
SPH & \verb|10| \\
\verb|10FCh| & \verb|05h| \\
\verb|10FDh| & \verb|03h| \\
\verb|10FEh| & \verb|18h| \\
\verb|10FFh| & \verb|00h| \\
\hline
\end{tabular}
\\

Wenn nun in Zeile 20 der \verb|RET|-Befehl augef�hrt wird, so wird die R�cksprungadresse vom Stack genommen und im Program Counter abgespeichert. Wir befinden uns also nach Ausf�hrung von Zeile 20 in Zeile 16.

Die Werte der Register bzw. Speicheradressen sind somit folgenderma�en:\\

\begin{tabular}{|lr|}
\hline
Register/Speicheradresse & Wert \\
\hline
PC & \verb|503h| \\
SPL & \verb|FE| \\
SPH & \verb|10| \\
\verb|10FEh| & \verb|18h| \\
\verb|10FFh| & \verb|00h| \\
\hline
\end{tabular}
\\

Bei der Ausf�hrung von Zeile 16 wird ein weiteres Mal die R�cksprungadresse vom Stack genommen, so dass wir als Werte in den Registern nun bekommen:\\

\begin{tabular}{|lr|}
\hline
Register/Speicheradresse & Wert \\
\hline
PC & \verb|503h| \\
SPL & \verb|00| \\
SPH & \verb|11| \\
\hline
\end{tabular}

\section{Port-0-/Port-2-Komplex}

\subsection{Zyklische Abfrage und Ausgabe von Schaltern}

\subsubsection{Aufgabe}

Initialisieren Sie Port 0 als Ausgabe- und Port 2 als Eingabe-Port. Entwickeln Sie eine Software, die zyklisch die an Port 2 angeschlossenen Schalter abfragt und die Schaltzust"ande an Port 0 ausgibt.

\subsubsection{L"osung}

\lstinputlisting[caption={Zyklische Abfrage von Ausgabe von Schaltern},language=z8asm]{aufgabe32a.asm}

\subsection{Z"ahlen von Anzahl Schalter"anderungen (Zusatz-Aufgabe)}

\subsubsection{Aufgabe}

Es ist ein Programm zu entwickeln, welches die Anzahl "Anderungen der Schaltzust"ande an Port 2 z"ahlt und die aktuelle Anzahl "uber Port 0 ausgibt.

Die Abarbeitung soll sowohl im Einzelschrittmodus als auch im normalen Programmablauf erfolgen. Warum ergibt sich ein unterschiedliches Verhalten?

Verbessern Sie das Programm so, dass im normalen Programmablauf das erwartete Verhalten eintritt.

\subsubsection{L"osung}

\lstinputlisting[caption={Z"ahlen von Anzahl Schalter"anderungen},language=z8asm]{aufgabe32c.asm}

\section{Timer-1-Komplex}

\subsection{Generierung einer Impulsfolge mit 5 kHz}

\subsubsection{Aufgabe}

Initialisieren Sie Timer 0 (einschlie�lich Prescaler) derart, dass am Portausgang P36 (${T1}_{OUT}$) eine Impulsfolge mit einer Frequenz von 5 kHz geliefert wird (Modulo-N-Mode, OSC=8 MHz, Anlage 5). Porteingang P31 soll hierbei als Toreingang (${T1}_{IN}$) benutzt werden (Aktivierung: High-Pegel �ber entprellten Schalter S3, Anlage 1).

\subsubsection{L"osung}

TODO: Nico und Thomas: Inhalt einf"ugen

\subsection{Frequenzteiler}

\subsubsection{Aufgabe}

Entwickeln Sie eine Software, die den Timer 1 als Frequenzteiler f"ur eine an P31 (${T1}_{IN}$) anliegende kontinuierliche Frequenzfolge (1 MHz, Generator HM 8035) wirken l"asst. Weisen Sie den maximalen und minimalen Teilerfaktor nach!

\subsubsection{L"osung}

Warnung: diese L"osung ist unvollst"andig. R"ucksprache mit Denis erforderlich!!!

\lstinputlisting[caption={Frequenzteiler},language=z8asm]{aufgabe34b.asm}

\end{document}