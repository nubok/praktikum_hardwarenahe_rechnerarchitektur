\documentclass[10pt]{scrartcl}
\usepackage[latin1]{inputenc}
\usepackage[english, ngerman]{babel}
\usepackage[babel]{csquotes}
\usepackage{amsmath} % for eqref
\usepackage{tabularx}
\usepackage{pst-circ, pstricks-add}
\usepackage{graphicx}
\author{Denis Dietze \and Wolfgang Keller \and Nico Linke \and Thomas Schulte}

\title{Protokoll Versuch IT2}
\subtitle{Microcontroller Z8}
\begin{document}
\maketitle

\section{Port-1-Komplex}

\subsection{Konfiguration von Port 0 zur Speichererweiterung durch 256 Byte externen RAM}

\subsection{Adressen zum Zugriff auf den externen RAM}

Die Adressen des externen RAMs sind ab 4096 (\verb|0x1000|). % Ist dies korrekt?
Da 256 Byte externer RAM vorliegt, sind somit die Adressen des externen RAMs \verb|0x1000-0x10FF|.

\subsection{Befehle zum Zugriff auf den RAM}

\ldots % Inhalte einfuegen

\subsection{Teilaufgabe d} % TODO Bessere Ueberschrift

\begin{verbatim}
LD 248, 0x10 % P01M initialisieren
% wir wollen SP auf 0x1100 setzen, somit ist 
% SPL auf 0x00 auf SPH auf 0x11 zu setzen
LD 255, 0x00 % SPL
LD 254, 0x11
CALL 0x500
...
0x500: RET
\end{verbatim}

Die R�ckkehradresse liegt dann in der Adresse \verb|0x10FE| auf dem Stack, da der Program Counter 2 Bytes ben�tigt.

\section{Port-0-/Port 2-Komplex}

\section{Timer-0-Komplex/Pulsbreitenmodulator}

\section{Timer-1-Komplex}

\section{UAT (SIO)-Komplex}

\section{P33/P34/IRQ-1-Komplex}

\end{document}